% This is a sample file for the Undergraduate Faculty Program at
% PCMI, containing one lecture 
%
%   What is a Partial Differential Equation?
%
%  from the Park City Lectures of Andrew J. Bernoff.
%
% The sample file illustrates the use of epsf.tex to include postscript 
% graphics, as well as various "AmS-LaTeX" constructions from the amsmath
% package (which is automatically loaded by the pcms-l document class).
% To run this file you need the files 
%
%        pcms-l.cls and pcmslmod.tex
%


\documentclass[lecture,12pt,]{pcms-l}
\input pcmslmod.tex  % v.1.2
%\input epsf.tex
\usepackage[framed,numbered,autolinebreaks,useliterate]{mcode}
\usepackage{graphicx}
\usepackage[toc,page]{appendix}
%\usepackage{natbib}
%\usepackage{amssymb} % this command would have loaded all the extra symbols,



% authors should not define these, they will be defined by the volume editors
%\def\currentvolume{3}
%\def\currentyear{1993}


% EQUATION NUMBERING AND THEOREM SETUP

\numberwithin{section}{chapter}
\numberwithin{equation}{chapter}

\theoremstyle{plain}
\newtheorem{theorem}[equation]{Theorem}
\newtheorem{lemma}[equation]{Lemma}

\theoremstyle{definition}
\newtheorem{example}{Example}[section]

\theoremstyle{definition}
\newtheorem{definition}[equation]{Definition}

\newtheorem{exercise}{Exercise}
\newtheorem{problem}{Problem}
\newtheorem*{remark}{Remark}

% Set enumerate to use letters, not numbers for problem parts.

\renewcommand{\theenumi}{\alph{enumi}}
\renewcommand{\labelenumi}{(\theenumi)}

% AUTHOR-DEFINED MACROS:

\newcommand{\cA}{\mathcal{A}}
\newcommand{\cB}{\mathcal{B}}
\newcommand{\C}{\mathbb{C}}
\newcommand{\cC}{\mathcal{C}}
\newcommand{\cD}{\mathcal{D}}

\begin{document}

\mainmatter
\setcounter{page}{1}

%\LogoOn

\lectureseries[Bayesian Analysis]{Decision Theory and Bayesian Analysis}


\auth[Dr. Vilda Purutcuoglu]{Dr. Vilda Purutcuoglu}
\address{Tubitak Space Technologies Research Institute, Ankara, 06800, Turkey}

\footnote{Edited by Anil A. Aksu based on lecture notes of STAT 565 course by Dr. Vilda Purutcuoglu }
%\email{ajb@hmc.edu}



% the following hack starts the lecture numbering at 1
\setcounter{lecture}{0}
\setcounter{chapter}{0}
\tableofcontents

\lecture{Bayesian Paradigm}

%\section{Outline of Lecture}
%\begin{itemize}
%\item{What is a Partial Differential Equation?}
%\item{Classifying PDE's: Order, Linear vs. Nonlinear}
%\item{Homogeneous PDE's and Superposition}
%\item{The Transport Equation} 
%\end{itemize}


\section{Bayes theorem for distributions}

If $A$ and $B$ are two events,
\begin{equation}
P(A\mid B)=\frac{P(A)P(B\mid A)}{P(B)}.
\end{equation}
This is just a direct consequence of the multiplication law of probabilities that says we can express $P(A\mid B)$ as either $P(A)P(B\mid A)$ or $P(B)P(A\mid B)$. For discrete distributions, if $Z,Y$ are discrete random variables
\begin{equation}
\label{eq:1}
P(Z=z\mid Y = y)=\frac{P(Z =z)P(Y = y\mid Z =z)}{P(Y = y)}.
\end{equation}
\begin{itemize}
\item How many distributions do we deal with here?
\end{itemize}
We can express the denominator in terms of the distribution in the numerator\cite{Allen}.
\begin{equation}
P(Y = y)=\sum_z P(Y = y, Z =z)=\sum_z P(Z= z)P(Y = y\mid Z =z).
\end{equation}
\begin{itemize}
\item This is sometimes called the law of total probability 
\end{itemize}
In this context, it is just an expression of the fact that as $z$ ranges over the possible values of $Z$
, the probabilities on the left hand-side of equation \ref{eq:1} make up the distribution of $Z$ given $Y=y$, and so they must add up to one. The extension to continuous distribution is easy. If $Z,Y$ are continuous random variable,
\begin{equation}
f(Z\mid Y)=\frac{f(Z)f(Y\mid Z)}{f(Y)}.
\end{equation}
where the denominator is now expressed as an integral:
\begin{equation}
f(Y)=\int f(Z)f(Y\mid Z)\mathrm{d}Z.
\end{equation}
\begin{equation}
f=
\left\{\begin{matrix}
continous \quad name ?\\ 
discrete \quad name ?
\end{matrix}\right.
\end{equation}


\vfill
%\begin{center}
%{\it (Room for notes)}
%\end{center}
\eject

\section{How Bayesian Statistics Uses Bayes Theorem}
\begin{theorem}[Bayes' theorem]
\label{Bayes}
\[P(A\mid B)=\frac{P(A)P(B\mid A)}{P(B)}\]
$P(B)$=if we are interested in the event $B$, $P(B)$ is the initial or prior probability of the occurence of event $B$. Then we observe event $A$ 
\\
$P(B\mid A)=$ How likely $B$ is when $A$ is known to have occurred is the posterior probability $P(B\mid A)$. 
\end{theorem}
Bayes' theorem can be understood as a formula for updating from prior to posterior probability, the updating consists of multiplying by the ratio $P(B\mid A)/P(A)$. It describes how a probability changes as we learn new information. Observing the occurrence of $A$ will increase the probability of $B$ if $P(B\mid A)>P(A)$.
From the law of total probability,
\begin{equation}
P(A)=P(A\mid B)P(B)+P(A\mid B^c)+P(A\mid B^c)P(B^c).
\end{equation}
where $P(B^c)=1-P(B)$.
\begin{lemma}
\[P(A\mid B)-P(A)=\frac{P(A)-P(A\mid B^c)P(B^c)}{1-P(B^c)}-P(A)\]
\end{lemma}
 
\begin{proof}
\[P(A\mid B)-P(A)=\frac{P(A)-P(A\mid B^c)P(B^c)-P(A)+P(A)P(B^c)}{P(B)}\]
\[P(A\mid B)-P(A)=\frac{P(B^c)(P(A)-P(A\mid B^c))}{P(B)}\]
\[P(A\mid B)-P(A)=P(B^c)(\frac{P(B)P(A\mid B)+P(B^c)P(A\mid B^c)}{P(B)}-\frac{P(A\mid B^c)}{P(B)})\]
\[P(A\mid B)-P(A)=P(B^c)(P(A\mid B)-\frac{P(A\mid B^c)(1-P(B^c))}{P(B)})\]
\[P(A\mid B)-P(A)=P(B^c)(P(A\mid B)-P(A\mid B^c))\]
\end{proof}
\subsection{Generalization of the Bayes' Theorem}
Let $B_1,...,B_n$ be a set of mutually exclusive events. Then
\begin{equation}
P(B_r\mid A)=\frac{P(B_r)P(A\mid B_r)}{P(A)}=\frac{P(B_r)P(A\mid B_r)}{\sum_{i=1}^n P(B_r)P(A\mid B_r)}.
\end{equation}
\begin{itemize}
\item Assuming that $P(B_r)>0$,$P(A\mid B)>P(A)$ if and only if $P(A\mid B)>P(A\mid B^c)$.
\item In Bayesian inference we use Bayes' theorem in a particular way.
\item $Z$ is the parameter (vector) $\theta$.
\item $Y$ is the data (vector) $X$.
\end{itemize}
So we have 
\begin{equation}
f(\theta\mid X)=\frac{f(\theta)f(X\mid \theta)}{f(X)}
\end{equation}
\begin{equation}
f(X)=\int f(\theta)f(X\mid \theta)\mathrm{d}\theta.
\end{equation}
\begin{equation}
f(\theta)=
\end{equation}
\begin{equation}
f(\theta\mid X)=
\end{equation}
\begin{equation}
f(X\mid \theta)=
\end{equation}
\subsection{Interpreting our sense} 
How do we interpret the things we see, hear, feel, taste or smell?
\begin{example}
I hear a song on the radio I identify the singer as Robbie Williams. Why do I think it's Robbie Williams?. Because he sounds like that. Formally, $P($ What I hear Robbie Williams $)>>P($What I hear someone else $)$
\end{example}
\begin{example}
I look out of the window and see what appears to be a tree. It has a big, dark coloured part sticking up out of the ground that branches into thinner sticks and on the ends of these are small green things. Clearly, $P($ What I hear Robbie Williams $)>>P($What I hear someone else $)$
\end{example}
\vfill
\eject

\section{Prior to Posterior}

\section{Triplot}

\footnote{All plots are generated in R, relevant codes are provided in Appendix R Codes }
\vfill
\eject

\subsection{Weak Prior Information}
It is the case where the prior information is much weaker that the data. This will occur, for instance, if we do not have strong information about $Q$ before seeing the  data, and if there are lots of data. Then in triplot, the prior distribution will be much broader and flatter that the likelihood. \underline{So the posterior is approximately proportional to the likelihood.}
\begin{example}
Classify the follow differential equations as ODE's or PDE's, linear
or nonlinear, and determine their order. For the linear equations,
determine whether or not they are homogeneous.
\end{example}



\vfill
\eject




\lecture{Some Common Probability Distributions}

\lecture{Inference}

\appendix


\chapter*{Basic Statistics}

\chapter*{R Codes}

\lstinputlisting[caption={Triplot Code in R}]{BasicStatistics.R}

\bibliographystyle{plain}
% Note the spaces between the initials
\bibliography{Bayes}
\end{document}