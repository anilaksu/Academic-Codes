
% Homework template for MA 614, Spring 2011.  When a line begins with the percent sign, the typesetter ignores it.  So, use percent signs at the beginning of lines to insert comments to yourself.


% Set the document class.  The command [11pt] sets the font at 11 point, which is nicer to read.  The default would be 10pt
\documentclass[11pt]{amsart} 


% Call packages that allow you to invoke certain mathematical symbols.
\usepackage{amssymb,amsmath,amsthm}
%\usepackage[framed,numbered,autolinebreaks,useliterate]{mcode}
\usepackage{graphicx}
%\usepackage{natbib}
% Set the title, author, and date information.
\title{EE 523: Take Home Midterm}
\author{Anil A. Aksu}
\date{\today}


% Formally begin the document and make the title.
\begin{document}
\maketitle

\section*{Part 1 }
A gyroelectric medium at frequency $\omega$ has the constitutive relations:
\\
$\bar{D}=\overline{\overline{\varepsilon}}\cdot \bar{E}$, $\bar{B}=\mu_0 \bar{H}$. The permeability $\mu$ is a scalar, whereas the dyadic permittivity $\overline{\overline{\varepsilon}}$ is represented by the matrix: 


\begin{equation*}
\overline{\overline{\varepsilon}} =
\begin{bmatrix}
 \varepsilon_1 & j \varepsilon_2 & 0 \\ 
-j \varepsilon_2 &  \varepsilon_1 & 0 \\ 
0 & 0 & \varepsilon_3
\end{bmatrix}
\end{equation*}
where $\varepsilon_1 > 0$, $\left |  \varepsilon_2 \right | < \varepsilon_1$, and $\varepsilon_3>0$.
\subsection*{a}
Our main aim is to characterize circularly polarized plane waves propagating in $z$-direction within this medium. For this purpose, we consider the two (complex-valued) unit vectors denoting the “circular” directions $\hat{e}_+ = \frac{1}{2}(\hat{a}_x-j\hat{a}_y)$ and $\hat{e}_- = \frac{1}{2}(\hat{a}_x + j \hat{a}_y)$ for right
and left polarizations. Show that any $\hat{E}$ field in the form $\hat{E}=E_x \hat{a}_x + E_y \hat{a}_y $ can be
converted to the form $\hat{E}=E_+ \hat{a}_+ + E_- \hat{a}_- $, where $E_+=E_x + j E_y$, and $E_-=E_x - j E_y$.
\\
Remark: Note that this operation is a basis change from the linear basis $\left \{ \hat{a}_x ,\hat{a}_y \right \}$ to the circular basis $\left \{ \hat{e}_+ ,\hat{e}_- \right \}$
\\
\textbf{Solution:}\\
The circular coordinate transformation defined in the problem can be given as:
\begin{equation}
\label{eq:1}
\begin{bmatrix}
 \frac{1}{2} & -\frac{j}{2} \\ 
\frac{1}{2} &  \frac{j}{2}
\end{bmatrix}
\begin{bmatrix}
 \hat{a}_x  \\ 
\hat{a}_y
\end{bmatrix}
=
\begin{bmatrix}
 \hat{e}_+ \\ 
\hat{e}_-
\end{bmatrix}
\end{equation}
And also,
\begin{equation}
\label{eq:2}
\begin{bmatrix}
 E_x &  E_y \\ 
\end{bmatrix}
\begin{bmatrix}
 \hat{a}_x  \\ 
\hat{a}_y
\end{bmatrix}
=
\begin{bmatrix}
 E_+ &  E_- \\ 
\end{bmatrix}
\begin{bmatrix}
 \hat{e}_+ \\ 
\hat{e}_-
\end{bmatrix}
\end{equation}
After replacing equation \ref{eq:1} into equation \ref{eq:2}, the following relation can be obtained:
\begin{equation}
\label{eq:3}
\begin{bmatrix}
 E_x &  E_y \\ 
\end{bmatrix}
\begin{bmatrix}
 \hat{a}_x  \\ 
\hat{a}_y
\end{bmatrix}
=
\begin{bmatrix}
 E_+ &  E_- \\ 
\end{bmatrix}
\begin{bmatrix}
 \frac{1}{2} & -\frac{j}{2} \\ 
\frac{1}{2} &  \frac{j}{2}
\end{bmatrix}
\begin{bmatrix}
 \hat{a}_x  \\ 
\hat{a}_y
\end{bmatrix}
\end{equation}
Therefore,
\begin{equation}
\label{eq:4}
\begin{bmatrix}
 E_x &  E_y \\ 
\end{bmatrix}
=
\begin{bmatrix}
 E_+ &  E_- \\ 
\end{bmatrix}
\begin{bmatrix}
 \frac{1}{2} & -\frac{j}{2} \\ 
\frac{1}{2} &  \frac{j}{2}
\end{bmatrix}
\end{equation}
After inverting the matrix in equation \ref{eq:4} and multiplying both side of equation \ref{eq:4}, the following system can be obtained:
\begin{equation}
\label{eq:5}
\begin{bmatrix}
 E_x &  E_y \\ 
\end{bmatrix}
\begin{bmatrix}
 1 & 1 \\ 
j &  -j
\end{bmatrix}
=
\begin{bmatrix}
 E_+ &  E_- \\ 
\end{bmatrix}
\end{equation}
As a result,
\begin{equation}
 E_+ = E_x + jE_y,
\end{equation}
\begin{equation}
 E_- = E_x - jE_y.
\end{equation}

\subsection*{b}
Show that under this transformation the permittivity matrix is diagonalized. That is:
\begin{equation*}
\begin{bmatrix}
 D_+\\ 
D_-\\ 
D_z
\end{bmatrix}
=
\begin{bmatrix}
 \varepsilon_1 + \varepsilon_2 & 0 & 0 \\ 
0 &  \varepsilon_1 -\varepsilon_2 & 0 \\ 
0 & 0 & \varepsilon_3
\end{bmatrix}
\begin{bmatrix}
E_+ \\ 
E_- \\ 
E_z 
\end{bmatrix}
\end{equation*}
where $D_{\pm}=D_x \pm j D_y$.
\\
\textbf{Solution:}\\
By indicial notation, the relation $\bar{D}=\overline{\overline{\varepsilon}}\cdot \bar{E}$ can be given as:
\begin{equation}
\label{eq:6}
D_i = \varepsilon_{ij}E_j.
\end{equation}
For given basis transformation, the linear coordinate transformation matrix is found as:
\begin{equation}
\label{eq:7}
a_{ij}=
\begin{bmatrix}
1 & j & 0 \\ 
1 &  -j & 0 \\ 
0 & 0 & 1
\end{bmatrix}
\end{equation}
After applying the coordinate transformation to the equation \ref{eq:6}, it can be given as:
\begin{equation}
\label{eq:8}
a_{ki}D_i=a_{ki}a_{lj}\varepsilon_{ij} a_{lj}E_j.
\end{equation}
Note that 
\begin{equation}
\label{eq:9}
a_{ki}D_i=
\begin{bmatrix}
 D_+\\ 
D_-\\ 
D_z
\end{bmatrix},
\end{equation}
\begin{equation}
\label{eq:10}
a_{lj}E_j=
\begin{bmatrix}
 E_+\\ 
E_-\\ 
E_z
\end{bmatrix},
\end{equation}
As a result,
\begin{equation}
a_{ki}a_{lj}\varepsilon_{ij}=
\begin{bmatrix}
1 & j & 0 \\ 
1 &  -j & 0 \\ 
0 & 0 & 1
\end{bmatrix}
\begin{bmatrix}
 \varepsilon_1 & j \varepsilon_2 & 0 \\ 
-j \varepsilon_2 &  \varepsilon_1 & 0 \\ 
0 & 0 & \varepsilon_3
\end{bmatrix}
\begin{bmatrix}
1 & 1 & 0 \\ 
j &  -j & 0 \\ 
0 & 0 & 1
\end{bmatrix}
=
\begin{bmatrix}
 \varepsilon_1 + \varepsilon_2 & 0 & 0 \\ 
0 &  \varepsilon_1 -\varepsilon_2 & 0 \\ 
0 & 0 & \varepsilon_3
\end{bmatrix}
\end{equation}

\subsection*{c}
Show that the wave vectors for right and left circularly polarized plane waves propagating
in z-direction are given as: $k_+ = \omega \sqrt{\mu \varepsilon_+}$ and $k_- = \omega \sqrt{\mu \varepsilon_-}$, where $\varepsilon_{\pm}=\varepsilon_1 \pm \varepsilon_2$. Note that the permittivity $\varepsilon_3$ has no
role in this formulation.
\\
\textbf{Solution:}\\
In a source free region, the governing equations for the electric field can be reduced to the following single differential equation: 
\begin{equation}
\label{eq:11}
(\mathbf{k}\cdot \mathbf{E})\mathbf{k}-(\mathbf{k} \cdot \mathbf{k})\mathbf{E}=-\omega^2 \mu_0\overline{\overline{\varepsilon}}\cdot \mathbf{E},
\end{equation}
and for the wave field $\bar{E}=E_0 e^{-j k_+ z}\hat{e}_+$, the equation \ref{eq:11} can be given as:
\begin{equation}
\label{eq:12}
-k_{+}^2 E_0 e^{-j k_+ z}=-\omega^2 \mu_0 \varepsilon_{+}E_0 e^{-j k_+ z}.
\end{equation}
For non-trivial electric field, the following relation must hold:
\begin{equation}
\label{eq:13}
k_+ = \omega \sqrt{\mu \varepsilon_+}
\end{equation}
Similarly, for the wave field $\bar{E}=E_0 e^{-j k_- z}\hat{e}_-$, the equation \ref{eq:11} can be given as:
\begin{equation}
\label{eq:14}
-k_{-}^2 E_0 e^{-j k_+ z}=-\omega^2 \mu_0 \varepsilon_{-}E_0 e^{-j k_+ z}.
\end{equation}
For non-trivial electric field, the following relation must hold:
\begin{equation}
\label{eq:15}
k_- = \omega \sqrt{\mu \varepsilon_-}
\end{equation}

\subsection*{d}
Assume that the region $0<z<d$ is filled with this gyroelectric medium, whereas the
regions $z<0$ and $z>d$ are free space. Assume that a linearly polarized plane wave $\bar{E}^i=E_0 e^{-j k_0 z}\hat{a}_x$ in the free space region $z<0$ is incident to the slab. Show that the transmitted wave in the region $z>d$ will be a linearly polarized plane wave $\bar{E}^t=\alpha E_0 \left [ \cos \psi \hat{a}_x + \sin \psi \hat{a}_y  \right ]e^{-j k_0 z}$, where $\alpha$ is a constant of proportionality (i.e. we are
not concerned with the reflected waves from the boundaries of the slab), and $\psi$ is the
angle of rotation of the polarization direction which depends on $k_+$, $k_-$ and $d$. Evaluate $\psi$.
\\
This phenomenon is known as Faraday rotation, and it was experimentally discovered by Michael Faraday in 1845.
\\
Hint: Decompose the wave propagating in the slab into its right and left circular
components, and relate the rotation angle to the phase difference between the circular
components at $z=d$, where the wave leaves the medium.
\\
\textbf{Solution:}\\

The interface conditions at $z=0$ can be given as:
\begin{equation}
E_{1t}=E_{2t},
\end{equation}
\begin{equation}
\label{eq:18}
\mathbf{a}_{n}\times(\mathbf{H}_{1}-\mathbf{H}_{2})=0.
\end{equation}
where $n$ and $t$ subscripts stand for normal and tangential components of the vector field. 

\begin{equation}
\bar{E}_{tra}=E_- e^{-j k_- z} \hat{e}_- + E_+ e^{-j k_+ z} \hat{e}_+ = T_{\bot}^1 E_0 e^{-j k_0 z}\hat{a}_x.
\end{equation}
which has to be satisfied at $z=0$ where $T_{\bot}^1$ is the transmission coefficients, therefore,
\begin{equation}
E_-=E_+= T_{\bot}^1 E_0.
\end{equation}
At $z=d$, the electric field can be given as:
\begin{equation}
T_{\bot}^1 E_0( e^{-j k_- d} \hat{e}_- + e^{-j k_+ d} \hat{e}_+)= \frac{T_{\bot}^1 E_0}{2} (\hat{a}_x(e^{-j k_- d}+e^{-j k_+ d})+j\hat{a}_y (e^{-j k_+ d}- e^{-j k_- d})).
\end{equation}
In the equation above, the sums and the differences of complex exponentials can be expanded as:
\begin{equation}
e^{-j k_- d}+e^{-j k_+ d}=e^{-j \frac{k_+ d+k_- d}{2}}(e^{-j \frac{k_+ d-k_- d}{2}}+e^{-j \frac{k_- d-k_+ d}{2}})=2 \cos (\frac{k_+ -k_- }{2}d)e^{-j \frac{k_+ +k_- }{2}d}
\end{equation}
and 
\begin{equation}
e^{-j k_+ d}-e^{-j k_- d}=e^{-j \frac{k_+ d+k_- d}{2}}(e^{-j \frac{k_+ d-k_- d}{2}}-e^{-j \frac{k_- d-k_+ d}{2}})=-2j \sin (\frac{k_+ -k_- }{2}d)e^{-j \frac{k_+ +k_- }{2}d}
\end{equation}
As a result, the field can be expressed as:
\begin{equation}
 T_{\bot}^1 E_0 e^{-j \frac{k_+ +k_- }{2}d}(\cos (\frac{k_+ -k_- }{2}d) \hat{a}_x+\sin (\frac{k_+ -k_- }{2}d)\hat{a}_y ).
\end{equation}
Since at the second transmission, the transmitted wave field after $z>d$ has to match the electric field above due  to interface conditions. The rotation angle of the polarization direction can be found as:
\begin{equation}
\psi=\frac{k_+ -k_- }{2}d.
\end{equation}
\newpage
\section*{Part 2 }
Using the guidelines given below, show that a plasma medium acts as a gyroelectric
medium when a constant external magnetic field $\bar{B}=B_0 \hat{a}_z$ is applied.

\subsection*{a}
The equation of motion of a free electron in this medium is governed by the following
differential equation $m_e \frac{\mathrm{d} \hat{v}}{\mathrm{d} t}=e(\hat{E}+\hat{v}\times \hat{B})$, where $m_e$ and $e$ are the electron mass and charge respectively. Show that the components of the velocity vector $\hat{v}$ satisfy:
\begin{equation*}
 \frac{\mathrm{d} v_x}{\mathrm{d} t}=\frac{e}{m_e} E_x+ \omega_b v_y,
\end{equation*}
\begin{equation*}
 \frac{\mathrm{d} v_y}{\mathrm{d} t}=\frac{e}{m_e} E_x - \omega_b v_x.
\end{equation*}
where $\omega_b=\frac{e B_0}{m_e}$ is called the cyclotron frequency.
\\
\textbf{Solution:}\\
The forcing due to magnetic field can be expanded as:
\begin{equation}
\label{eq:21}
\hat{v}\times \hat{B}=det(\begin{bmatrix}
\hat{a}_x & \hat{a}_y & \hat{a}_z \\ 
v_x &  v_y & 0 \\ 
0 & 0 & B_0
\end{bmatrix} )
=B_0 v_y \hat{a}_x - B_0 v_x \hat{a}_y.
\end{equation}
After replacing this term into equation of motion, the set of governing equations can be obtained as:
\begin{equation}
\label{eq:22}
m_e \frac{\mathrm{d} v_x}{\mathrm{d} t}=e (E_x+ B_0 v_y),
\end{equation}
\begin{equation}
\label{eq:23}
m_e \frac{\mathrm{d} v_y}{\mathrm{d} t}=e (E_y- B_0 v_x).
\end{equation}
Dividing both sides of equations \ref{eq:22} and \ref{eq:23} results in the following form:
\begin{equation}
\label{eq:24}
 \frac{\mathrm{d} v_x}{\mathrm{d} t}=\frac{e}{m_e} E_x+ \omega_b v_y,
\end{equation}
\begin{equation}
\label{eq:25}
 \frac{\mathrm{d} v_y}{\mathrm{d} t}=\frac{e}{m_e} E_y - \omega_b v_x.
\end{equation}
where $\omega_b=\frac{e B_0}{m_e}$ is called the cyclotron frequency.
\subsection*{b}
In order to solve these coupled differential equations, search (in phasor domain) solutions
in the form $v_{\pm}=v_x \pm j v_y$. Show that:
\begin{equation*}
v_{\pm}=v_x \pm j v_y=\frac{\frac{e}{m_e}(E_x \pm j E_y)}{j (\omega \pm \omega_b) }.
\end{equation*}
\\
\textbf{Solution:}\\
Fourier transforming equations \ref{eq:24} and \ref{eq:25} in time results in the following set of algebraic equations:
\begin{equation}
\label{eq:26}
j \omega v_x =\frac{e}{m_e} E_x+ \omega_b v_y,
\end{equation}
\begin{equation}
\label{eq:27}
j \omega v_y =\frac{e}{m_e} E_y - \omega_b v_x.
\end{equation}
In matrix form, equations \ref{eq:26} and \ref{eq:27} can be given as:
\begin{equation}
\label{eq:28}
\begin{bmatrix}
j \omega & -\omega_b \\  
\omega_b & j \omega
\end{bmatrix} 
\begin{bmatrix}
v_x  \\ 
 v_y
\end{bmatrix} 
=
\begin{bmatrix}
\frac{e}{m_e} E_x \\ 
\frac{e}{m_e} E_y
\end{bmatrix} 
\end{equation}
By inverting the matrix in equation \ref{eq:28}, the velocity components can be obtained as:
\begin{equation}
\label{eq:29}
\begin{bmatrix}
v_x  \\ 
 v_y
\end{bmatrix} 
=
\frac{1}{\omega_{b}^2 -\omega^2}
\begin{bmatrix}
j \omega & \omega_b \\  
-\omega_b & j \omega
\end{bmatrix} 
\begin{bmatrix}
\frac{e}{m_e} E_x \\ 
\frac{e}{m_e} E_y
\end{bmatrix} 
\end{equation}
Therefore,
\begin{equation}
\label{eq:30}
v_+ = v_x+ j v_y = \frac{\frac{e}{m_e}(E_x + j E_y)j(\omega-\omega_b)}{\omega_{b}^2 -\omega^2}=\frac{\frac{e}{m_e}(E_x + j E_y)}{j(\omega_{b} +\omega)},
\end{equation}
and also
\begin{equation}
\label{eq:31}
v_- = v_x- j v_y = \frac{\frac{e}{m_e}(E_x - j E_y)j(\omega+\omega_b)}{\omega_{b}^2 -\omega^2}=\frac{\frac{e}{m_e}(E_x - j E_y)}{j(\omega -\omega_b)}.
\end{equation}
\subsection*{c}
Finally let $N$ be the number of electrons per unit volume. Show that $\varepsilon_{\pm}=\varepsilon_1 \pm \varepsilon_2 = \varepsilon_0 \left [ 1 - \frac{\omega_{p}^2}{\omega (\omega \pm \omega_b)}  \right ]$, where $\omega_p = \sqrt{\frac{N e^2}{m_e \varepsilon_0}}$ is the plasma frequency.
\\
Hint: Integrate the velocity to find the displacement of a single electron and obtain the
susceptibility of the bulk material in phasor domain.
\\
\textbf{Solution:}\\
The dipole moment is defined as:
\begin{equation}
\label{eq:32}
\bar{p}(t)=q (-l(t)) \hat{e}{\pm}
\end{equation}
where $q=e$ and $l(t)$ can be obtained by integrating the velocity field $v_{\pm}$ as:
\begin{equation}
l(t)=\frac{v_{\pm}}{j \omega} = -\frac{\frac{e}{m_e}(E_x \pm j E_y)}{\omega (\omega \pm \omega_b)}
\end{equation}
The macroscopic electric polarization vector $\bar{P(t)}$ is evaluated as:
\begin{equation}
\bar{P}(t)=N \bar{p(t)}
\end{equation}
therefore,
\begin{equation}
\label{eq:33}
\bar{P}(t)= -\frac{\frac{N e^2}{m_e}}{\omega (\omega \pm \omega_b)}(E_x \pm j E_y).
\end{equation}
Also
\begin{equation}
\label{eq:34}
\bar{D}=\varepsilon_0 \bar{E}+\bar{P}.
\end{equation}
After replacing equation \ref{eq:33} into the relation \ref{eq:34}, the following relation is obtained:
\begin{equation}
\bar{D}=\varepsilon_0 \bar{E}(1-\frac{\frac{N e^2}{m_e \varepsilon_0}}{\omega (\omega \pm \omega_b)})
\end{equation}
As a result,
\begin{equation}
\varepsilon_{\pm}=\varepsilon_1 \pm \varepsilon_2 = \varepsilon_0 \left [ 1 - \frac{\omega_{p}^2}{\omega (\omega \pm \omega_b)}  \right ].
\end{equation}
\end{document}
