
% Homework template for MA 614, Spring 2011.  When a line begins with the percent sign, the typesetter ignores it.  So, use percent signs at the beginning of lines to insert comments to yourself.


% Set the document class.  The command [11pt] sets the font at 11 point, which is nicer to read.  The default would be 10pt
\documentclass[11pt]{amsart} 


% Call packages that allow you to invoke certain mathematical symbols.
\usepackage{amssymb,amsmath,amsthm}


% Set the title, author, and date information.
\title{Homework 3}
\author{Anil Aksu}
\date{\today}


% Formally begin the document and make the title.
\begin{document}
\maketitle

\section*{Problem 1 }

Compute the sectional curvature in Poincar\'{e} Plane.
\\
\textbf{Solution:}\\
Poincar\'{e} metric $g$ on a manifold $M$ is given as:
\begin{equation}
g=\frac{\mathrm{d}x\otimes \mathrm{d}x +\mathrm{d}y\otimes \mathrm{d}y }{y^2}.
\end{equation} 
And also the sectional curvature is defined as:
\begin{equation}
K(u,v)=-\frac{<R(u,w)u,w>}{<u,u><v,v>-<u,v>^2}.
\end{equation} 
where
\begin{equation*}
R_{\beta jk}^{\alpha}=\frac{\partial \Gamma_{\beta k}^{\alpha}}{\partial x_j}-\frac{\partial \Gamma_{\beta j}^{\alpha}}{\partial x_k}+ \Gamma_{\gamma j}^{\alpha} \Gamma_{\beta k}^{\gamma}- \Gamma_{\gamma k}^{\alpha} \Gamma_{\beta j}^{\gamma}.
\end{equation*}
Christoffel  symbol of second kind $\Gamma_{\beta j}^{\gamma}$ can be computed using Euler-Lagrange equation. First, let's define the energy functional as:
\begin{equation}
\label{eq:1}
\int_{a}^{b}F(\vec{\dot{x}},\vec{x},t)\mathrm{d}t=\int_{a}^{b}\dot{x}^{i}\dot{x}^{j}g_{ij}\mathrm{d}t.
\end{equation}
The energy functional above has to satisfy Euler-Lagrange equations which are given as:
\begin{equation}
\label{eq:2}
\frac{\mathrm{d}}{\mathrm{d} t}(\frac{\partial F}{\partial \dot{x}^i})-\frac{\partial F}{\partial x_i}=0.
\end{equation}
Equation \ref{eq:2}  will result in the following set of differential equations:
\begin{equation}
\label{eq:3}
\frac{\mathrm{d}^2 x^i}{\mathrm{d} t^2}+\Gamma_{j k}^{i}\dot{x}^j \dot{x}^k =0.
\end{equation}
In Poincar\'{e} Plane, the energy functional is defined as:
\begin{equation}
\label{eq:4}
F(\vec{\dot{x}},\vec{x},t)=\frac{\dot{x}^2+\dot{y}^2}{y^2}.
\end{equation}
Therefore, Euler-Lagrange equations give the following set of ordinary differential equations:
\begin{equation}
\label{eq:5}
\ddot{x}-2\frac{\dot{x}\dot{y}}{y}=0,
\end{equation}
\begin{equation}
\label{eq:5}
\ddot{y}-\frac{\dot{y}^2}{y}+\frac{\dot{x}^2}{y}=0.
\end{equation}
Therefore, Christoffel symbols are obtained as:
\begin{equation}
\Gamma_{x y}^{x}=\Gamma_{y x}^{x}=-\frac{1}{y},
\end{equation}
\begin{equation}
\Gamma_{y y}^{y}=-\frac{1}{y},
\end{equation}
\begin{equation}
\Gamma_{x x}^{y}=\frac{1}{y}.
\end{equation}
By using Christoffel symbols derived above, let's write down the components of  the curvature $R_{\beta jk}^{\alpha}$,
\begin{equation}
R_{xxx}^{x}=\frac{\partial \Gamma_{x x}^{x}}{\partial x}-\frac{\partial \Gamma_{x x}^{x}}{\partial x}+ \Gamma_{\gamma x}^{x} \Gamma_{x x}^{\gamma}- \Gamma_{\gamma x}^{x} \Gamma_{x x}^{\gamma}=0,
\end{equation}
and
\begin{equation}
R_{xxy}^{x}=\frac{\partial \Gamma_{x y}^{x}}{\partial x}-\frac{\partial \Gamma_{x x}^{x}}{\partial y}+ \Gamma_{\gamma x}^{x} \Gamma_{x y}^{\gamma}- \Gamma_{\gamma y}^{x} \Gamma_{x x}^{\gamma}=-R_{xyx}^{x}=0,
\end{equation}
and
\begin{equation}
R_{xyy}^{x}=\frac{\partial \Gamma_{x y}^{x}}{\partial y}-\frac{\partial \Gamma_{x y}^{x}}{\partial y}+ \Gamma_{\gamma y}^{x} \Gamma_{x y}^{\gamma}- \Gamma_{\gamma y}^{x} \Gamma_{x y}^{\gamma}=0,
\end{equation}
and
\begin{equation}
R_{yxx}^{x}=\frac{\partial \Gamma_{y x}^{x}}{\partial x}-\frac{\partial \Gamma_{y x}^{x}}{\partial x}+ \Gamma_{\gamma x}^{x} \Gamma_{y x}^{\gamma}- \Gamma_{\gamma x}^{x} \Gamma_{y x}^{\gamma}=0,
\end{equation}
and
\begin{equation}
R_{yxy}^{x}=\frac{\partial \Gamma_{y y}^{x}}{\partial x}-\frac{\partial \Gamma_{y x}^{x}}{\partial y}+ \Gamma_{\gamma x}^{x} \Gamma_{y y}^{\gamma}- \Gamma_{\gamma y}^{x} \Gamma_{y x}^{\gamma}=-R_{yyx}^{x}=-\frac{1}{y^2},
\end{equation}
and
\begin{equation}
R_{yyy}^{x}=\frac{\partial \Gamma_{y y}^{x}}{\partial y}-\frac{\partial \Gamma_{y y}^{x}}{\partial y}+ \Gamma_{\gamma y}^{x} \Gamma_{y y}^{\gamma}- \Gamma_{\gamma y}^{x} \Gamma_{y y}^{\gamma}=0,
\end{equation}
Furthermore,
\begin{equation}
R_{xxx}^{y}=\frac{\partial \Gamma_{x x}^{y}}{\partial x}-\frac{\partial \Gamma_{x x}^{y}}{\partial x}+ \Gamma_{\gamma x}^{y} \Gamma_{x x}^{\gamma}- \Gamma_{\gamma x}^{y} \Gamma_{x x}^{\gamma}=0,
\end{equation}
and
\begin{equation}
R_{xxy}^{y}=\frac{\partial \Gamma_{x y}^{y}}{\partial x}-\frac{\partial \Gamma_{x x}^{y}}{\partial y}+ \Gamma_{\gamma x}^{y} \Gamma_{x y}^{\gamma}- \Gamma_{\gamma y}^{y} \Gamma_{x x}^{\gamma}=-R_{xyx}^{y}=\frac{1}{y^2},
\end{equation}
and
\begin{equation}
R_{xyy}^{y}=\frac{\partial \Gamma_{x y}^{y}}{\partial y}-\frac{\partial \Gamma_{x y}^{y}}{\partial y}+ \Gamma_{\gamma y}^{y} \Gamma_{x y}^{\gamma}- \Gamma_{\gamma y}^{y} \Gamma_{x y}^{\gamma}=0,
\end{equation}
and
\begin{equation}
R_{yxx}^{y}=\frac{\partial \Gamma_{y x}^{y}}{\partial x}-\frac{\partial \Gamma_{y x}^{y}}{\partial x}+ \Gamma_{\gamma x}^{y} \Gamma_{y x}^{\gamma}- \Gamma_{\gamma x}^{y} \Gamma_{y x}^{\gamma}=0,
\end{equation}
and
\begin{equation}
R_{yxy}^{y}=\frac{\partial \Gamma_{y y}^{y}}{\partial x}-\frac{\partial \Gamma_{y x}^{y}}{\partial y}+ \Gamma_{\gamma x}^{y} \Gamma_{y y}^{\gamma}- \Gamma_{\gamma y}^{y} \Gamma_{y x}^{\gamma}=-R_{yyx}^{y}=0,
\end{equation}
and
\begin{equation}
R_{yyy}^{y}=\frac{\partial \Gamma_{y y}^{y}}{\partial y}-\frac{\partial \Gamma_{y y}^{y}}{\partial y}+ \Gamma_{\gamma y}^{y} \Gamma_{y y}^{\gamma}- \Gamma_{\gamma y}^{y} \Gamma_{y y}^{\gamma}=0.
\end{equation}
After expanding curvature $R_{\beta jk}^{\alpha}$, let's write down vectors $u$ and $v$ explicitly,
\begin{equation}
u=a_1 \frac{\partial}{\partial x}+a_2 \frac{\partial}{\partial y},
\end{equation}
and 
\begin{equation}
v=b_1 \frac{\partial}{\partial x}+b_2 \frac{\partial}{\partial y}.
\end{equation}
As a result,
\begin{equation}
R(u,v)u=(b_1 a_2-a_1 b_2)\frac{a_2}{y^2}\frac{\partial}{\partial x}+(a_1 b_2-b_1 a_2)\frac{a_1}{y^2}\frac{\partial}{\partial y}
\end{equation}
Therefore,
\begin{equation}
<R(u,v)u,v>=\frac{(a_1 b_2-b_1 a_2)^2}{y^4}.
\end{equation}
and also
\begin{equation}
<u,u><v,v>-<u,v>^2=\frac{(a_{1}^2+a_{2}^2)(b_{1}^2+b_{2}^2)- (a_1 b_1+ a_2 b_2)^2}{y^4}=\frac{(a_1 b_2-b_1 a_2)^2}{y^4}
\end{equation}
Finally,
\begin{equation}
K(u,v)=-1.
\end{equation} 
\end{document}


